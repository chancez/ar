\documentclass[letterpaper,10pt]{article}

\usepackage{graphicx}

\usepackage{amssymb}
\usepackage{amsmath}
\usepackage{amsthm}

\usepackage{alltt}
\usepackage{float}
\usepackage{color}

\usepackage{url}

\usepackage{balance}
\usepackage[TABBOTCAP, tight]{subfigure}
\usepackage{enumitem}

\usepackage{pstricks, pst-node}

\usepackage{geometry}
\geometry{textheight=10in, textwidth=7.5in}

\usepackage{hyperref}

\def\name{Chance Zibolski}

%% The following metadata will show up in the PDF properties
\hypersetup{
  colorlinks = true,
  urlcolor = black,
  pdfauthor = {\name},
  pdfkeywords = {cs311 ``operating systems'' assignment week1},
  pdftitle = {CS 311 Assignment 1},
  pdfsubject = {CS 311 Assignment 1},
  pdfpagemode = UseNone
}

\parindent = 0.0 in
\parskip = 0.2 in

\title{CS311 Assignment 1}
\author{\name}
\date{October 2013}

\begin{document}

\maketitle

\section{Design}

For the ar program I plan on separating out each key into a function, where all
the logic for creating an archive will just be a composition of re-usable
functions.

There are two types of functions needed for AR. The first is a function which
accepts the list of file names passed to the program, which will be used for
options -q, -x, -d, -t, which are all used with a list of file names on the
command line. The other options -A and -v will be acting slightly different,
where the -A will be passed a list of files, but it wont be a list from the
command line, and -v will be working with an existing flag, which will likely
deal with disabling suppressing output of commands.

With this in mind, almost all of my ar options will be acting on a list of file
names, so I will be designing my functions to work this way. I'll begin with -q
so I can create an archive and test its output. The rest should simply deal
with reading the ar data structure and parsing the ar header data. That should
also be trivial. I will be able to read text from the archive until reaching
the ARMAG constants, which will let me iteratively go through the file and
either print the -t contents, delete files using -d or extract files using -x.
All of those will be very similar, so I will likely have utility functions for
reading the file until the ARMAG constants and parsing the ar header data.

\end{document}
