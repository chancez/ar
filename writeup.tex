\documentclass[letterpaper,10pt]{article}

\usepackage{graphicx}
\usepackage{longtable}

\usepackage{amssymb}
\usepackage{amsmath}
\usepackage{amsthm}

\usepackage{alltt}
\usepackage{float}
\usepackage{color}

\usepackage{url}

\usepackage{balance}
\usepackage[TABBOTCAP, tight]{subfigure}
\usepackage{enumitem}

\usepackage{pstricks, pst-node}

\usepackage{geometry}
\geometry{textheight=10in, textwidth=7.5in}

\usepackage{hyperref}

\newcommand{\longtableendfoot}{Continued on next page}

\def\name{Chance Zibolski}

%% The following metadata will show up in the PDF properties
\hypersetup{
  colorlinks = true,
  urlcolor = black,
  pdfauthor = {\name},
  pdfkeywords = {cs311 ``operating systems'' assignment week1},
  pdftitle = {CS 311 Assignment 1},
  pdfsubject = {CS 311 Assignment 1},
  pdfpagemode = UseNone
}

\parindent = 0.0 in
\parskip = 0.2 in

\title{CS311 Assignment 2}
\author{\name}
\date{October 2013}

\begin{document}

\maketitle

\section{Design}

For the ar program I plan on separating out each key into a function, where all
the logic for creating an archive will just be a composition of re-usable
functions.

There are two types of functions needed for AR. The first is a function which
accepts the list of file names passed to the program, which will be used for
options -q, -x, -d, -t, which are all used with a list of file names on the
command line. The other options -A and -v will be acting slightly different,
where the -A will be passed a list of files, but it wont be a list from the
command line, and -v will be working with an existing flag, which will likely
deal with disabling suppressing output of commands.

With this in mind, almost all of my ar options will be acting on a list of file
names, so I will be designing my functions to work this way. I'll begin with -q
so I can create an archive and test its output. The rest should simply deal
with reading the ar data structure and parsing the ar header data. That should
also be trivial. I will be able to read text from the archive until reaching
the ARMAG constants, which will let me iteratively go through the file and
either print the -t contents, delete files using -d or extract files using -x.
All of those will be very similar, so I will likely have utility functions for
reading the file until the ARMAG constants and parsing the ar header data.

\subsection{Deviations from Original Design}
I pretty much followed my plan from beginning to end. The only part that I
really deviated with was where I found I could not extract the looping through
the archive file logic. It was mostly because I wouldn't be able to simply
store all of that data into memory and had to deal with things as I iterated
through the archive file, and this was simply a limitation of the design.

Since extraction, deletion, printing the table of contents, all required
iterating through the archive file, and we could only pass through it at most
one time, I ended up having to use a case statement inside the loop. I could
have used call backs which all returned how far they iterated through the file,
however I disliked the idea of that and went with the case statement instead.

\section{Work Log}
% To be filled in when finished
%% This file was generated by the script latex-git-log
%% Base git commit URL: https://github.com/ecnahc515/ar/commit
\begin{tabular}{lp{12cm}}
  \label{tabular:legend:git-log}
  \textbf{acronym} & \textbf{meaning} \\
  V & \texttt{version} \\
  tag & \texttt{git tag} \\
  MF & Number of \texttt{modified files}. \\
  AL & Number of \texttt{added lines}. \\
  DL & Number of \texttt{deleted lines}. \\
\end{tabular}

\bigskip
\begin{longtable}{|rllp{12cm}rrr|}
\hline \multicolumn{1}{|c}{\textbf{V}} & \multicolumn{1}{c}{\textbf{tag}}
& \multicolumn{1}{c}{\textbf{date}}
& \multicolumn{1}{c}{\textbf{commit message}} & \multicolumn{1}{c}{\textbf{MF}}
& \multicolumn{1}{c}{\textbf{AL}} & \multicolumn{1}{c|}{\textbf{DL}} \\ \hline
\endhead

\hline \multicolumn{7}{|r|}{\longtableendfoot} \\ \hline
\endfoot

\hline% \hline
\endlastfoot

\hline 1 &  & 2013-10-22 & \href{https://github.com/ecnahc515/ar/commit/2af6c9fe6d950574b135e501a05924a80bfff867}{Inital commit} & 2 & 90 & 0 \\
\hline 2 &  & 2013-10-29 & \href{https://github.com/ecnahc515/ar/commit/f2c03e805482a5d7702df773ca7688b2407a655b}{Add gitignore and license} & 3 & 46 & 0 \\
\hline 3 &  & 2013-10-22 & \href{https://github.com/ecnahc515/ar/commit/24d1ff0ff9b5bf0b81194e627022c8a99e8c519d}{More planning on writeup} & 1 & 23 & 1 \\
\hline 4 &  & 2013-10-22 & \href{https://github.com/ecnahc515/ar/commit/d975e701138799aa01e14c4bc079e82a85da2e10}{Add getopt working example from gnu.org} & 1 & 54 & 0 \\
\hline 5 &  & 2013-10-22 & \href{https://github.com/ecnahc515/ar/commit/df0ecfb0775bda7f67bc439a4783db6f837c487a}{Update makefile} & 1 & 9 & 5 \\
\hline 6 &  & 2013-10-22 & \href{https://github.com/ecnahc515/ar/commit/e8e5694f6d0351ae347a9e54e4d2721c34fe26ff}{Add -std=c99} & 2 & 44 & 20 \\
\hline 7 &  & 2013-10-22 & \href{https://github.com/ecnahc515/ar/commit/bcfdd48a7e95349f6f7beeded9fbe4629c415c94}{Working file read} & 1 & 28 & 2 \\
\hline 8 &  & 2013-10-23 & \href{https://github.com/ecnahc515/ar/commit/31e1059d22ace3d1feb5e39fcdb3b6fcc64f0ac7}{Creation and opening with -q properly working.} & 1 & 57 & 9 \\
\hline 9 &  & 2013-10-23 & \href{https://github.com/ecnahc515/ar/commit/87ad1b658494e3539aa0a257db3a52a155105912}{We can open with append instead of using fcntl to adjust it later} & 1 & 2 & 10 \\
\hline 10 &  & 2013-10-23 & \href{https://github.com/ecnahc515/ar/commit/79617277e887c0552626e1550491bc9a48dc13e0}{myar now properly writes ar\_hdr data structs to a file.} & 2 & 87 & 29 \\
\hline 11 &  & 2013-10-23 & \href{https://github.com/ecnahc515/ar/commit/4d0ab1e956010b7310821fbe0cb01ae5e65a6ddf}{Add file\_stat pretty printing} & 2 & 38 & 0 \\
\hline 12 &  & 2013-10-23 & \href{https://github.com/ecnahc515/ar/commit/97b866bbde0aa9c3746ec656a09129ef8f54a1ff}{Appending file contents to archive works} & 1 & 31 & 22 \\
\hline 13 &  & 2013-10-23 & \href{https://github.com/ecnahc515/ar/commit/e9f22116d678cbcdeba495a074599b21623d018a}{Begin error handling of -t} & 1 & 30 & 2 \\
\hline 14 &  & 2013-10-23 & \href{https://github.com/ecnahc515/ar/commit/6d422ef992be6c5da391016e5dc23f6ce57ba814}{Fix writing header. Format string was off by one} & 1 & 5 & 4 \\
\hline 15 &  & 2013-10-28 & \href{https://github.com/ecnahc515/ar/commit/738b57df5134b4d9e4c316f8e5097df54840b776}{Better buffer size.} & 1 & 32 & 6 \\
\hline 16 &  & 2013-10-28 & \href{https://github.com/ecnahc515/ar/commit/c3332e1ca9e67dde72d29d8e222327b6492b724b}{Better function names} & 1 & 12 & 12 \\
\hline 17 &  & 2013-10-28 & \href{https://github.com/ecnahc515/ar/commit/fb3c7333898a9d000140b724ae93dafb0dff0207}{No need for malloc here} & 1 & 17 & 13 \\
\hline 18 &  & 2013-10-28 & \href{https://github.com/ecnahc515/ar/commit/9375b513c653eca1679f28f6f69275134a367825}{Working non-verbose -t arg} & 1 & 24 & 17 \\
\hline 19 &  & 2013-10-28 & \href{https://github.com/ecnahc515/ar/commit/f312e2d30415da2cf962841ef869e19811569573}{Replicating file for -x mostly done} & 1 & 88 & 8 \\
\hline 20 &  & 2013-10-28 & \href{https://github.com/ecnahc515/ar/commit/407dde6e7a14244538b95d3f221156189ee53637}{Extract now working correctly on file names. Checks if the filename is passed in the args now.} & 1 & 72 & 33 \\
\hline 21 &  & 2013-10-29 & \href{https://github.com/ecnahc515/ar/commit/2c3cf01da2544b8bdfe8c3e24dc721ac55481444}{Extract works, multiple files, no args, and alignments are accounted for} & 1 & 45 & 15 \\
\hline 22 &  & 2013-10-29 & \href{https://github.com/ecnahc515/ar/commit/af032903a476d4ed0676d9253184238254850f4b}{Debug target for makefile} & 1 & 5 & 3 \\
\hline 23 &  & 2013-10-29 & \href{https://github.com/ecnahc515/ar/commit/916af40f8f78b2ed03c9d9774ec23bc46f548a85}{Poorly planned commit. Moved stuff around and delete working.} & 1 & 225 & 146 \\
\hline 24 &  & 2013-10-29 & \href{https://github.com/ecnahc515/ar/commit/2469b9e4094874d7a02dce38c7911a50e0f27b9a}{Remove unused code} & 1 & 0 & 5 \\
\hline 25 &  & 2013-10-29 & \href{https://github.com/ecnahc515/ar/commit/b7d03cbd0bd5000a2c49c8edc7f8353a576e8d86}{Only use basename of file} & 1 & 9 & 3 \\
\hline 26 &  & 2013-10-29 & \href{https://github.com/ecnahc515/ar/commit/9a104b26492d3f638b06c3682bff4d6a23a071f0}{Unlink archive if run into non reg file} & 1 & 5 & 2 \\
\hline 27 &  & 2013-10-29 & \href{https://github.com/ecnahc515/ar/commit/d2b39fd55a961b9b04e80d7c3d742a5a16e71d9f}{Set mtime on file.} & 1 & 0 & 2 \\
\hline 28 &  & 2013-10-29 & \href{https://github.com/ecnahc515/ar/commit/0ae44765426e08bd1fe7097eaa53b895ea734d59}{Print message if it creates the archive} & 1 & 1 & 0 \\
\hline 29 &  & 2013-10-29 & \href{https://github.com/ecnahc515/ar/commit/d6b99f1cc8f91c5229d3843c93f3c5ea9b83476c}{Verbose -t working} & 1 & 35 & 17 \\
\hline 30 &  & 2013-10-29 & \href{https://github.com/ecnahc515/ar/commit/381812ea94df111dd8068b6ba33dd728b9fc4ed6}{Working verbose -q} & 1 & 4 & 2 \\
\hline 31 &  & 2013-10-29 & \href{https://github.com/ecnahc515/ar/commit/6a8cf521aff52b8b53aeea6a1a38046387263f61}{Verbose -x working} & 1 & 10 & 7 \\
\hline 32 &  & 2013-10-29 & \href{https://github.com/ecnahc515/ar/commit/f2068def438b0aea7696799e0b8ab921b5c61e5e}{Whoops forgot myar.h} & 1 & 31 & 0 \\
\hline 33 &  & 2013-10-30 & \href{https://github.com/ecnahc515/ar/commit/1faba77db8bc54b191cde66fcc5adedf0ce2c9f1}{Extract appending of single file. Append all working} & 2 & 90 & 28 \\
\hline 34 &  & 2013-10-30 & \href{https://github.com/ecnahc515/ar/commit/718ad07fbb86a3f0bc6953968c348339a89c3918}{Better error checking and messages} & 2 & 32 & 10 \\
\hline 35 &  & 2013-10-30 & \href{https://github.com/ecnahc515/ar/commit/64355f21adb5b047900dcb3913772ee0c0bc9ad5}{Begin challenges on writeup} & 1 & 9 & 0 \\
\hline 36 &  & 2013-10-30 & \href{https://github.com/ecnahc515/ar/commit/ac799d5bac24949bf3a93e7c146a696ef30425f4}{Answer questions to writeup} & 1 & 60 & 1 \\
\hline 37 &  & 2013-10-30 & \href{https://github.com/ecnahc515/ar/commit/4b0f5f075dafd993e2e6163702a2ea3d6817b35d}{Add latex-git-log for generating table} & 1 & 362 & 0 \\
\hline 38 &  & 2013-10-30 & \href{https://github.com/ecnahc515/ar/commit/c61caf39bc6614ba3252bf00da487363c61e67c5}{remove german check from latex-git-log} & 1 & 1 & 3 \\
\hline 39 &  & 2013-10-30 & \href{https://github.com/ecnahc515/ar/commit/d5f4f0b2dd48f9b517860c930897b166c6726a8c}{Add work log} & 2 & 66 & 0 \\
\hline 40 &  & 2013-10-30 & \href{https://github.com/ecnahc515/ar/commit/f23ba1843cc9fa3ad5bdd47fa638e3611b56dc3e}{Ignore tex output files} & 1 & 6 & 0 \\
\hline 41 &  & 2013-10-30 & \href{https://github.com/ecnahc515/ar/commit/c319db7ccbf9d58a08c96eb1cca574b6c67ecbda}{Update worklog} & 1 & 2 & 0 \\
\hline 42 &  & 2013-10-30 & \href{https://github.com/ecnahc515/ar/commit/f4d2322e82b44ae00f7326a25d83abeba7e6c4d9}{Update writeup with design deviations} & 1 & 14 & 1 \\
\hline 43 &  & 2013-10-30 & \href{https://github.com/ecnahc515/ar/commit/c64aae4eeacf7cd6160721294a620efd6522e15e}{update work log} & 1 & 2 & 0 \\
\hline 44 &  & 2013-10-30 & \href{https://github.com/ecnahc515/ar/commit/3ea2c3ad416b8f4c20fa00fa6212f6c5c66a4aef}{Only mention creating new archive if not deleting} & 2 & 7 & 6 \\
\hline 45 &  & 2013-10-30 & \href{https://github.com/ecnahc515/ar/commit/d8c26fcbc2b5606f1622b9beee57db94ee76e44b}{Fix byte alignment issues on deletion} & 2 & 46 & 12 \\
\hline 46 &  & 2013-10-30 & \href{https://github.com/ecnahc515/ar/commit/6ab8293fda922318f3deb8a7f5bbf8f3a0f693b6}{Add verbose to delete} & 2 & 5 & 3 \\
\hline 47 &  & 2013-10-30 & \href{https://github.com/ecnahc515/ar/commit/3651c07710c29fa8a85830bb6e9c2de4e89c39e1}{Only delete/extract first occurence of file.} & 1 & 18 & 4 \\
\hline 48 &  & 2013-10-30 & \href{https://github.com/ecnahc515/ar/commit/1018614bb966ff7ccabf9a91015e57f99e191979}{Check for at least one arg} & 1 & 9 & 8 \\
\end{longtable}


\section{Challenges}
The assignment itself was fairly straight forward, and implementation itself
wasn't too difficult either. My biggest challenge was getting stuck on the
small off by one errors. In particular when trying to get extraction working, I
wasn't checking for if read() returned 0 for end of file. Along with that, I
was accounting for even byte boundaries in most places, but when working on the
extraction piece I forgot about that as well, and things were not working due
to that.

So overall the biggest challenge was debugging. Tools like GDB helped a little,
but most of my problems were due to lack a lack of familiarity with particular
functions and system calls.

\section{Questions}
\begin{description}
\item[Q1:] What do you think the main point of this assignment is?
\item[A1:] I think the point of this assignment is to introduce the class to a
slightly complex C application with a couple of different components to worry
about. Since we will likely be using command line args for most of our
programs, learning to use getopt in C one of the points. I think its main
purpose was to become familiar with the system call API.

\item[Q2:] How did you ensure your solution was correct? Testing details, for instance.
\item[A2:] When testing I built archives using the built in unix ar program,
with various sizes (even and odd), different owners, groups, ect. I then tested
replicating these archive files with my archiver and compared them using diff
to ensure consistency.

Next I would use my archiver to extract, delete, and print items from the real
archive file created with the built in ar, and I would use the built in ar to
do the same on the archive file created with my archiver. I would then use diff
to compare all results with the original files, as well as the ones extracted
from the various archiver programs.

To test the -A option, I created test directories with various file types
inside of them, and tested that it didn't archive itself or the executable.
When testing verbose output, I would redirect the output from the real ar into
a file, and then do the same with mine, and compare the files using diff again.

\item[Q3:] What did you learn?
\item[A3:] I learned that in order to do well on these assignments you need to
work very iteratively and make sure you understand the underlying problem. You
also need to become very familiar with the tools your using for the given task.
As I mentioned in my challenges, reading the man pages a little more carefully
would have saved me a lot of time running into issues.

I also found that you can't sit too long staring at a single problem, and that
if your stuck, to try finding something else you can work on until you've got a
fresh mind to take a look at the old problem you were having. In on case, I
could not get utime to work for some reason and stopped on it for a while.
After finishing the rest of another command, I went back to fix utime and it
was working because of another bug I fixed along the way.

Finally I learned that asking for help isn't a terrible thing. Sometimes it's
easy to forget small things that you understand, but aren't perceiving
correctly, and someone from the outside might be able to immediately pick up on
what your doing wrong. For example when trying to set file permissions back
onto the file, I was getting the wrong permissions set. With help from a friend
realized my issue was that I wasn't setting the mode back to octal, and that
was my issue.
\end{description}

\end{document}
